\documentclass{article}
\usepackage[utf8]{inputenc}
\usepackage[russian,english]{babel}
\usepackage[calc,useregional,showdow]{datetime2}
\usepackage{titletoc}
\dottedcontents{section}[1.5in]{\bfseries\large}{1.5in}{10pt}

\usepackage{xparse}
\NewDocumentCommand{\datesec}{o m m}{%
    \renewcommand\thesection{\DTMdate{#2}}
    \IfNoValueTF{#1}
        {\section{#3}}
        {\section[#1]{#3}}
}
\usepackage{color}   %May be necessary if you want to color links
\usepackage{hyperref}
\hypersetup{
    colorlinks=true, %set true if you want colored links
    linktoc=all,     %set to all if you want both sections and subsections linked
    linkcolor=blue,  %choose some color if you want links to stand out
}
\begin{document}
\selectlanguage{english}
  \section{Intro}
% Begin with explaining what work is being done. Also, who is involved in the project? State these parties. This will lead to a standing offer, which cements prices for products or services purchased for the project, and a more formal contract that goes into greater detail.
  \section{ What Is the Purpose of the Project}
% Start with the big question: why are you initiating this project? What’s the purpose of doing the project? Create a purpose statement to lead off this section and provide a thorough answer to these questions, such as what are the deliverables, objectives and return on investment.
  \section{ Scope of Work}
% What work needs to be done in the project? Note it here, including what hardware and software will be necessary. What is the process you’ll use to complete the work? This includes outcomes, time involved and even general steps it’ll take to achieve that.
  \section{ Where Will the Work Be Done}
% The team you employ will have to work somewhere. The project might be site specific, at a central facility or some, if not all, the team members could work remotely. Either way, here is where you want to detail that and where the equipment and software used will be located.
  \section{ Tasks}
% Take those general steps outlined in the scope of work and break them down into more detailed tasks. Be specific and don’t leave out any action that is required of the project to produce its deliverables. If you want to, break the tasks down into milestones or phases.
  \section{ Milestones}
% Define the amount of time that is scheduled to complete the project, from the start date to the proposed finish date. Detail the billable hours per week and month, and whatever else relates to the scheduling of the project. Again, specificity counts. For example, if there’s a maximum amount of billable hours for vendors and/or contracts, note it here.
  \section{ Deliverables}
% What are the deliverables of the project? List them and explain what is due and when it is due. Describe them in detail, such as quantity, size, color and whatever might be relevant.
  \section{ Schedule}
% Include a detailed list of when the deliverables need to get done, beginning with which vendor will be selected to achieve this goal, the kickoff, what the period of performance is, the review stage, development, implementation, testing, close of the project, etc.
  \section{ Standards and Testing}
% If there are any industry standards that need to be adhered to, list those here. Also, if there will be testing of the product, list who will be involved in this process, what equipment is needed and other resources.
  \section{ Define Success}
% Note what the sponsor and/or stakeholder expects as a successful project completion.
  \section{ Requirements}
% List any other equipment that is needed to complete the project and if there is a necessary degree or certification required of team members. Also, note if there will be travel or other aspects of the project not already covered.
  \section{ Payments}
% If the budget has been created, then you can list the payments related to the project, and how they will be delivered, up front, over time or after completion. For example, you can pay after the completion of a milestone or on a fixed schedule, whichever is more financially feasible.
  \section{ Other}
% There will be other parts of the project that are not suited to the above categories, and this is the place where you can add them so that everything is covered. For example, are there security issues, restrictions around hardware or software, travel pay, post-project support, etc?
  \section{ Closure}
% This will determine how the deliverables will be accepted, and who will deliver, review and sign off on the deliverables. Also, it deals with the final admin duties, making sure everything is signed and closed and archived.

\end{document}
