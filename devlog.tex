\documentclass{article}
\usepackage[english]{babel}
\usepackage[calc,useregional,showdow]{datetime2}
\usepackage{titletoc}
\dottedcontents{section}[1.5in]{\bfseries\large}{1.5in}{10pt}

\usepackage{xparse}
\NewDocumentCommand{\datesec}{o m m}{%
    \renewcommand\thesection{\DTMdate{#2}}
    \IfNoValueTF{#1}
        {\section{#3}}
        {\section[#1]{#3}}
}
\begin{document}
\begin{titlepage}
   \begin{center}
       \vspace*{1cm}

       \textbf{DevLog}

       \vspace{0.5cm}
        Devlog for PyLearnGTK

       \vspace{1.5cm}

       \textbf{Arseni Yamnii}

       \vfill


       \vspace{0.8cm}



       StackOverflow\\
       Git\\
       Linux\\
       Python3\\
       LaTex\\
       GTK\\
       2020

   \end{center}
\end{titlepage}
\tableofcontents
\newpage
\datesec{2020-04-29}{create DevLog}
here il starting write dev log with all my thinks(misly in russian) about this project
\newpage
\datesec{2020-05-02}{LaTex}
Start using Latex
And so... I create my template for DevLog. It is like diary for application. Here i write some tips and triks)).\\
I think sometimes i create script to integrate this tis to Documentation. Maby with some Tags, etc.\\
Yes, it is long project... with UML and texts. But it pretty shit functionals. BUT mayby sometimes it get more functional, to got more popularity.\\
TODO:
\begin{itemize}
  \item Create UML diagram with TikzUML. Yes in latex.....
  \item Find examples to python with GTK
  \item Create GTK UI
  \item create tests to program
\end{itemize}
\datesec{2020-05-03}{UML day one}
Few days ago i find latex package TIKZ-UML. this package was created for create UML diagrams in latex. I think it is the best tool for create uml, because for transfer source to another achine, you only need to transfer few files. .tex file, and library tikz-uml. And then compile to format what you want.\\
TODO:
\begin{itemize}
  \item crete folder for code examples
  \item look throgt tikz-uml example code, and create own UML diagrams for project. I dont know why, but for new expierance
\end{itemize}
\end{document}

