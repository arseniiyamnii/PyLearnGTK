\documentclass{article}
\usepackage[utf8]{inputenc}
\usepackage[russian,english]{babel}
\usepackage[calc,useregional,showdow]{datetime2}
\usepackage{titletoc}
\dottedcontents{section}[1.5in]{\bfseries\large}{1.5in}{10pt}

\usepackage{xparse}
\NewDocumentCommand{\datesec}{o m m}{%
    \renewcommand\thesection{\DTMdate{#2}}
    \IfNoValueTF{#1}
        {\section{#3}}
        {\section[#1]{#3}}
}
\usepackage{color}   %May be necessary if you want to color links
\usepackage{hyperref}
\hypersetup{
    colorlinks=true, %set true if you want colored links
    linktoc=all,     %set to all if you want both sections and subsections linked
    linkcolor=blue,  %choose some color if you want links to stand out
}
\begin{document}
\begin{titlepage}
   \begin{center}
       \vspace*{1cm}

       \textbf{DevLog}

       \vspace{0.5cm}
        Devlog for PyLearnGTK

       \vspace{1.5cm}

       \textbf{Arseni Yamnii}

       \vfill


       \vspace{0.8cm}



       StackOverflow\\
       Git\\
       Linux\\
       Python3\\
       LaTex\\
       GTK\\
       2020

   \end{center}
\end{titlepage}
\tableofcontents
\section{PreDeveloping}
\newpage
\datesec{2020-04-29}{create DevLog}
here il starting write dev log with all my thinks(misly in russian) about this project
\newpage
\datesec{2020-05-02}{LaTex}
Start using Latex
And so... I create my template for DevLog. It is like diary for application. Here i write some tips and triks)).\\
I think sometimes i create script to integrate this tis to Documentation. Maby with some Tags, etc.\\
Yes, it is long project... with UML and texts. But it pretty shit functionals. BUT mayby sometimes it get more functional, to got more popularity.\\
TODO:
\begin{itemize}
  \item Create UML diagram with TikzUML. Yes in latex.....
  \item Find examples to python with GTK
  \item Create GTK UI
  \item create tests to program
\end{itemize}
\newpage
\datesec{2020-05-03}{UML day one}
Few days ago i find latex package TIKZ-UML. this package was created for create UML diagrams in latex. I think it is the best tool for create uml, because for transfer source to another achine, you only need to transfer few files. .tex file, and library tikz-uml. And then compile to format what you want.\\
TODO:
\begin{itemize}
  \item crete folder for code examples
  \item look throgt tikz-uml example code, and create own UML diagrams for project. I dont know why, but for new expierance
\end{itemize}
\newpage
\datesec{2020-05-05}{Little tis about UML}
UML got 15 diagrams :\\
\begin{itemize}
  \item class - DONE
  \item component
  \item composite structure
  \item collaboration
  \item deployment
  \item object
  \item package 
  \item profile
  \item activity
  \item state machine
  \item use case
  \item communication
  \item interaction overview
  \item sequence
  \item timing
\end{itemize}
I need learn all about this 15 diagrams, and create exampples, or working diagrams to my project. For new expiarence.
\datesec{2020-05-07}{UML class diagram}
I create i think good class diagram to first try. When i learn GTK, and UML more, i recreate it diagram. Tikz-uml goood))
\datesec{2020-05-08}{Ready with UM-class, and note about Sphinx}
Few day ago, i see, that Linux kernel used Sphinx as auto-documentation tool. I need learn more about sphinx
UPD: Right now i undestand, that i dont use Software Development Cycle. I need that.
\begin{itemize}
  \item plannning
  \item analysis
  \item design
  \item implementation
  \item testing and integration
  \item maintenance
\end{itemize}
\datesec{2020-05-10}{Planning}
Planning it is Scope of Work (SOW). It is just few quastions, and i must answer to them. But! What is the questions? I just gogle "scope of work questions", and find many questions for SOW. I ust read them all (only fro first page of google), and create fram them all y questions. Soething like Universal SOW questions. Every queston ust be unic.
\datesec{2020-05-11}{Scope/Statement of work}
Scope of work questions (in english and russian):
\selectlanguage{russian}
русский
\selectlanguage{english}
sorry. i write it in english. But now i know how to write in many languages in one latex file
\datesec{2020-05-13}{Scope/Statement of work:Part 2}
Scope of work:
\begin{itemize}
  \item Deliverables\\
    what we doing. all stuff. Product/service. All things.
  \item Milestones\\
    Major project stages
  \item Timeline\\
    project timeline. We can use Gantt chart
  \item Reports\\
    Whoch reprts we can do from all project stages
\end{itemize}
It is from \url{https://www.projectmanager.com/training/write-scope-work} \\
Oh fuck. It is just part of Statement of work. Statement of work i learn from \url{https://www.projectmanager.com/blog/statement-work-definition-examples}
Statement:
\begin{itemize}
  \item Intro:\\ Begin with explaining what work is being done. Also, who is involved in the project? State these parties. This will lead to a standing offer, which cements prices for products or services purchased for the project, and a more formal contract that goes into greater detail.
  \item  What Is the Purpose of the Project:\\ Start with the big question: why are you initiating this project? What’s the purpose of doing the project? Create a purpose statement to lead off this section and provide a thorough answer to these questions, such as what are the deliverables, objectives and return on investment.
  \item  Scope of Work:\\ What work needs to be done in the project? Note it here, including what hardware and software will be necessary. What is the process you’ll use to complete the work? This includes outcomes, time involved and even general steps it’ll take to achieve that.
  \item  Where Will the Work Be Done:\\ The team you employ will have to work somewhere. The project might be site specific, at a central facility or some, if not all, the team members could work remotely. Either way, here is where you want to detail that and where the equipment and software used will be located.
  \item  Tasks:\\ Take those general steps outlined in the scope of work and break them down into more detailed tasks. Be specific and don’t leave out any action that is required of the project to produce its deliverables. If you want to, break the tasks down into milestones or phases.
  \item  Milestones:\\ Define the amount of time that is scheduled to complete the project, from the start date to the proposed finish date. Detail the billable hours per week and month, and whatever else relates to the scheduling of the project. Again, specificity counts. For example, if there’s a maximum amount of billable hours for vendors and/or contracts, note it here.
  \item  Deliverables:\\ What are the deliverables of the project? List them and explain what is due and when it is due. Describe them in detail, such as quantity, size, color and whatever might be relevant.
  \item  Schedule:\\ Include a detailed list of when the deliverables need to get done, beginning with which vendor will be selected to achieve this goal, the kickoff, what the period of performance is, the review stage, development, implementation, testing, close of the project, etc.
  \item  Standards and Testing:\\ If there are any industry standards that need to be adhered to, list those here. Also, if there will be testing of the product, list who will be involved in this process, what equipment is needed and other resources.
  \item  Define Success:\\ Note what the sponsor and/or stakeholder expects as a successful project completion.
  \item  Requirements:\\ List any other equipment that is needed to complete the project and if there is a necessary degree or certification required of team members. Also, note if there will be travel or other aspects of the project not already covered.
  \item  Payments:\\ If the budget has been created, then you can list the payments related to the project, and how they will be delivered, up front, over time or after completion. For example, you can pay after the completion of a milestone or on a fixed schedule, whichever is more financially feasible.
  \item  Other:\\ There will be other parts of the project that are not suited to the above categories, and this is the place where you can add them so that everything is covered. For example, are there security issues, restrictions around hardware or software, travel pay, post-project support, etc?
  \item  Closure:\\ This will determine how the deliverables will be accepted, and who will deliver, review and sign off on the deliverables. Also, it deals with the final admin duties, making sure everything is signed and closed and archived.
\end{itemize}
\end{document}

